\section{Introduction}

\begin{center}
\textit{\LARGE ``Un robot n'est pas tout à fait une machine. Un robot est une machine fabriquée pour imiter de son mieux l'être humain.``} 

\huge Isaac Asimov, \textit{La cité des robots}
\end{center}

\vspace{0.8cm}
Dans le cadre de ma Licence 3 Sciences Pour l'Ingénieur parcours \'Electronique, \'Energie électrique et Automatique j'ai réalisé un stage de 8 semaines au sein de l'association 1901 \textit{Robots !}. J'ai souhaité faire mon stage au sein de cette association car j'ai pour projet professionnel de travailler en tant qu'ingénieur biomédical dans le domaine de la conception de prothèses robotisées destinées aux personnes souffrant de handicap. Bien que ce domaine ne soit pas directement lié aux différents projets encadrés par l'association, j'ai notamment été séduit par les différents projets qu'elle entreprend. Ici, le robot n'est pas le centre de l'attention mais c'est bien l'humain qui apprend à utiliser le robot pour améliorer son quotidien.


\vspace{0.5cm}
Pour intégrer l'équipe tout en me rapprochant de mon objectif professionnel et en restant dans l'esprit de l'association, Sophie Sakka m'a proposé de mettre en place un projet spécifique : un atelier de fabrication et de programmation d'une prothèse robotisée. Toutefois, la contrainte du temps mais aussi le fait que mon atelier devait pouvoir s'adresser au grand public m'ont fait pencher pour une main robotisée plutôt qu'une prothèse. L'atelier que j'ai mis en place avait donc pour but de sensibiliser les personnes à la \textbf{CAO} (ou \textbf{CAD} en anglais), à l'impression \textbf{3D}, à l'électronique et à la programmation.


\vspace{0.5cm}
Durant ces 8 semaines, j'ai mis en place un atelier de fabrication de main robotisée accessible à tous. Au delà de la fabrication du prototype de main, la finalité était que je puisse animer cet atelier durant ma dernière semaine de stage. La durée de cet atelier était de 4 demi-journées réparties du lundi 22 Février au jeudi 25 Février.


\vspace{0.5cm}
Dans un premier temps je rappellerai l'histoire de l'association. En seconde partie, j'aborderai les différentes missions que j'ai eu à réaliser pour mettre en place mon atelier. Enfin, je conlurai sur ce que ce stage m'a apporté et j'évoquerai différentes perspectives d'améliorations envisageables pour ce projet.