\newpage

\section{Conclusion}

J'ai eu la chance de mener au cours de ce stage un projet en accord avec mon objectif professionnel. Le fait que Sophie Sakka m'ait proposé de créer cet atelier et de l'encadrer pour l'association m'a permis de découvrir différentes facettes de la gestion de projet, tout en étant encadré et aidé par des personnes dans la réalisation de ces différentes tâches.

\vspace{0.5cm}
Tout cela m'a aidé à visualiser ce qu'étaient les contraintes auxquelles nous devons faire face pour mener à bien un projet dans un délai imparti. Dans ce contexte j'ai réussi à aller au bout de l'objectif, à savoir animer l'atelier durant la dernière semaine. Les 4 personnes inscrites ont toutes pu repartir avec leur propre main et avec, je l'espère, l'envie d'améliorer le projet que j'avais mis en place ou de se lancer seuls dans de nouvelles conceptions robotiques.


\vspace{0.5cm}
Le projet initial prévoyait de pousser la partie électronique un peu plus loin en portant le microcontroleur de l'Arduino \textit{Uno}, les composants nécessaires de la carte et les potentiomètres sur une plaque de \textbf{PCB}. Pour des raisons de complexité, de coût mais également de sécurité, le projet a été revu. En effet, cela incluait d'appprendre aussi aux participants à faire de la soudure et nécessitait donc du matériel et des compétences supplémentaires.

\vspace{0.5cm}
J'ai aussi pris conscience qu'il faut toujours prévoir une marge de manoeuvre lors de la réalisation d'un projet. Certaines tâches peuvent s'avérer plus longues que ce que l'on avait imaginé au départ et il est important de réaliser un planning dès le début. Il peut aussi parfois y avoir des tâches qui prennent du retard sans que cela soit forcément de notre fait (impressions ratées, réception tardives des commandes, etc...) et là encore le fait d'avoir anticipé en prévoyant une date butoir un peu plus loin que nécessaire peut s'avérer utile !

\vspace{0.5cm}
J'ai pu acquérir de nouvelles compétences en me livrant à un exercice que je n'avais encore jamais fait, à savoir réaliser un projet et développer des connaissances dans le but de les transmettre à un public, et lui permettre de réaliser le montage et la programmation de la main de façon autonome. 

\vspace{0.5cm}
Dans les perspectives d'améliorations, il est possible d'envisager un progrès esthétique en remplaçant l'avant bras inclus dans le projet sur lequel je me suis basé par un socle fermé duquel ne ressortiraient que les potentiomètres. Cette nouvelle dimension du projet aurait nécessité de passer plus de temps pour faire de la \textbf{CAO}, conduisant ainsi à un dépassement de la durée du stage.

\vspace{0.5cm}
À mon sens, la gestion de projet est une compétence très importante lorsque l'on est ingénieur. \'Etant donné que ceci fait partie du projet professionnel auquel j'aspire, je pense que cela a été une chance énorme d'avoir pu experimenter cela pendant ce stage.

\vspace{0.5cm}
Ce stage m'a enfin permis d'appronfondir également mes connaissances dans le domaine de l'impression \textbf{3D}.



