\section{Performance}

\subsection{Quelle utilité pour cette main robotique?}
\subsubsection{Exprimer son humeur et/ou un mot}

\begin{flushleft}
En effet, il est possible via l'utilisation des potentiomètres de régler les positions de chaque doigt de manière indépendante. Il est donc possible d'exprimer son humeur via des gestes simples du quotidien.\vspace{0.2cm}

\textbf{Quelques exemples :}

Pour faire comme les plongeurs et signaler que tout va bien, il suffit de faire former au Pouce et à l'Index un rond. Par conséquent il suffit de bouger les potentiomètres respectifs de chacun de ces doigts pour réaliser cette forme avec votre main !

Pour signaler un énervement, on peut imaginer qu'un poing fermé serait la bonne traduction de cette humeur. Il suffit donc de tourner au maximum chaque potentiomètre afin que tous les doigts se referment sur la paume de la main.

Pour exprimer une réussite, on peut former le V de la victoire avec l'Index et le majeur. Il suffit donc de rabattre le Pouce, l'Annulaire et l'Auriculaire sur la paume de la main.

\`A vous de trouver d'autres expressions à faire dire à votre main !

\subsubsection{Dire un mot ou une phrase en langue des signes}

Munissez vous d'un dictionnaire de la langue des signes sur internet, et bougez les potentiomètres de sorte à réaliser une expression proche de celle du mot que vous voulez dire. La langue des signes ne se repose que sur les mouvements de la main mais aussi sur un mouvement de bras et une expression de visage? Mouvez votre bras robotique avec votre propre main pour réaliser ces mouvements, ou bien trouvez votre propre manière de les réaliser !

\subsubsection{Jouer à \textit{"Pierre, Papier, Ciseaux"}}

Si vous souhaitez construire ce bras pour jouer à ce jeu, utilisez trois boutons poussoirs plutôt que des potentiomètres, chaque bouton que vous connecterez à la main correspondra à une des trois positions ! Imaginez et rédigez votre propre programme pour commander les différents mouvements.

\end{flushleft}

%provisoire
\newpage

\section{Pour aller plus loin}

\subsection{Comment créer son propre modèle en CAO ?}

\begin{flushleft}
    L’étape suivante va être de modéliser vous même les pièces que vous souhaitez imprimer. Pour cela il faudra vous munir d’un logiciel de CAO. Dans la partie d’introduction à la CAO vous trouverez un lien renvoyant vers une liste de logiciels de CAO, pour la plupart étant gratuits ou possédant une version gratuite.
    
    Pour modéliser vos pièces il vous faut un logiciel, mais vous devez surtout définir les dimensions et la forme de la pièce que vous vouler concevoir. Pour cela il faut prendre en compte certaines contraintes comme l’utilité qu’aura la pièce (Un objet de décoration, une pièce mécanique, une pièce destinée à recevoir en son sein un (ou des) composant(s) électronique(s), etc...), la taille qu'elle devra avoir, etc...
    
    Pour prendre en main les logiciels et de manière générale la CAO, il existe de nombreux tutoriels sur YouTube qui permettent de s'initier étapes par étapes.
\end{flushleft}

\subsection{Comment choisir ses composants électroniques ?}

\begin{flushleft}
    Pour choisir vos composants il est nécessaire de vous renseigner sur leurs dimensions (que votre modèle 3D soit adapté pour accueillir les composants choisis). Il faut également se renseigner sur les caractéristiques techniques (datasheet en anglais), afin de s'assurer qu'elles soient en accord avec l'utilisation souhaitée.
    
    Tout cela se trouve généralement sur le site du fabricant ou le site d'achat du composant. Sinon tapez "datasheet" suivi du nom du composant sur internet.
\end{flushleft}


\subsection{Perspectives d'amélioration}

\begin{flushleft}
    Si vous souhaitez pousser encore plus loin votre expérience, vous pouvez envisager de rendre votre main un peu plus indépendante de votre Arduino en portant votre carte sur une breadboard ou, encore plus ambitieux, vous lancer dans la soudure et la porter sur une plaque de PCB. 
    
    Il est également possible d'essayer de vous rapprocher encore plus du projet de \href{https://www.instructables.com/3D-Printed-Robotic-Hand/}{TechMartian} en essayant de contrôler la main par Bluetooth.
    
    \begin{multicols}{2}
    \includegraphics[width=60pt,height=60pt]{Déroulé/Jour_1/Manuel d'utilisation/Images/6.jpg}
    
    \columnbreak
    
\textbf{\large Attention : }\textbf{\textit{En plus des composants         nécessaires mentionnés en figure \ref{2.1}, il vous faudra investir     dans un module Bluetooth compatible avec la carte Arduino car nous     n'utilisons pas la même que TechMartian, et il faudra également        modifier le programme Arduino.}}\\
\end{multicols}

Il est également envisageable de décider de modéliser son propre robot via un logiciel, pour concevoir un robot complet. De nouvelles contraintes se présenteront alors, différentes de celles que nous avons rencontré lors de ce projet.

Nous n'avons ici présenté que certains exemples d'améliorations à apporter, mais il en existe sans doute bien d'autres ! \`A vous de réfléchir à celles qui vous intéressent et à la manière dont vous souhaitez les mettre en place. 
\end{flushleft}

